\documentclass{article}
\usepackage[utf8]{inputenc}
\usepackage[spanish]{babel}
\usepackage{listings}
\usepackage{graphicx}
\graphicspath{ {images/} }
\usepackage{cite}

\begin{document}

\begin{titlepage}
    \begin{center}
        \vspace*{1cm}
            
        \Huge
        \textbf{Manejo eficiente de la memoria}
            
        \vspace{0.8 cm}
        \LARGE
        Taller Memoria Informática II
            
        \vspace{1.5cm}
            
        \textbf{Juan Pablo Guerra Callejas}
        
        \vspace{0.8 cm}
        
        \textbf{Docentes: Augusto Salazar\\
                Jonathan Gómez}
        \vfill
            
        \vspace{0.8cm}
            
        \Large
        Despartamento de Ingeniería Electrónica y Telecomunicaciones\\
        Universidad de Antioquia\\
        Medellín\\
        Septiembre de 2020
            
    \end{center}
\end{titlepage}

\tableofcontents
\newpage
\section{Funcionalidad de la memoria en un computador}\label{intro}


El avance tecnológico ha ido fuertemente ligado a la evolución de los dispositivos electrónicos, un ejemplo de estos dispositivos es el computador. El computador es una máquina digital programable con la que el usuario interactúa utilizando una serie de comandos para procesar datos de entrada con el fin de obtener un resultado o información a la salida. La mayoría de las personas que se relacionan con el mundo de la tecnología son capaces de discernir entre el software y el hardware del computador, lo realmente complicado es profundizar en las partes más importante que lo componen. Entre estas partes se encuentra un componente que parece conceptualmente sencillo, pero abarca la mayor diversidad de tipos, tecnología y coste en comparación con los demás componentes del computador. \\

Tal y como se definió el computador es posible hacerse una idea del papel tan primordial que presenta la memoria. Esta, es el dispositivo donde se almacena temporalmente toda la información con la que trabajan los microprocesadores. Se podría decir que la memoria es un intermediario entre la información almacenada en el disco duro y el procesamiento de información que realizan los microprocesadores. Si bien es cierto que el procesador puede utilizar el disco duro como la fuente directa para el procesado, este sería muchísimo más lento, debido principalmente a la distancia que se encuentra del núcleo del microprocesador.
Una vez el sistema deja de utilizar el contenido extraído del disco duro y almacenado en celdas de memoria más cercanas, se elimina de dichos espacios, de este modo es posible utilizar la capacidad de la memoria en otras instrucciones del procesador y así priorizar la velocidad de ejecución.\\

En resumen, la memoria se utiliza principalmente para agilizar el procesamiento de información en un computador, almacenando el contenido extraído del disco duro con la ayuda del microprocesador hacia un espacio o dirección de memoria más cercano a los núcleos.


\section{Tipos de memoria conocidos} \label{contenido}
La memoria, tal y como se comentó en el punto anterior alberga un sinfín de diseños y arquitecturas. Por esto, los usuarios reducen su clasificación en tres cuestiones: Velocidad, Capacidad y sobretodo coste; todas ligeramente ligadas entre ellas.\\

A continuación se presentan diferentes tipos de memoria, organizadas en cuestión de mayor a menor velocidad y por lo tanto, de menor a mayor capacidad.

\subsection{Memoria ROM (Read Only Memory)}

La memoria ROM o "read only memory" se utiliza únicamente durante el encendido del computador. La característica que mejor la diferencia de las demás es que no presenta información de forma volátil, es decir, nunca se borra el contenido o comandos que se encuentran almacenados en ella. La función que tiene la ROM es la de proporcionar al procesador la instrucción para chequear el funcionamiento de los componentes del sistema

\subsection{Memoria Caché (L1, L2, L3)}

La memoria caché se posiciona en el lugar número uno en cuanto a velocidad y cercanía de los núcleos del procesador. El objetivo de la memoria caché es almacenar la información recurrente utilizada por el usuario, con el fin de aumentar la velocidad de la memoria a la hora de procesar los datos ya que se evita ingresar en el contenido del disco duro y de la memoria RAM.\\

La caché se caracteriza por mantener diferentes niveles de memoria: L1, L2, L3.
L1 es el nivel más rápido y con menor capacidad. La velocidad se debe a su cercanía con el núcleo del procesador ya que se encuentra dentro del mismo.
L2 sigue siendo una memoria rápida, aunque en menor medida a L1 sí se puede afirmar que presenta una mayor capacidad.
En L3 se almacena el resto de información "cacheada", tomando la cronología anterior se puede decir que esta memoria es más lenta y tiene mayor capacidad que las dos anteriores.

\subsection{Memoria RAM (Random Acces Memory)}



\subsection{Disco duro}


\section{Gestión de memoria}

\section{Conclusiones}


\bibliographystyle{IEEEtran}
\bibliography{references}

\end{document}
