\documentclass{article}
\usepackage[utf8]{inputenc}
\usepackage[spanish]{babel}
\usepackage{listings}
\usepackage{graphicx}
\graphicspath{ {images/} }
\usepackage{cite}

\begin{document}

\begin{titlepage}
    \begin{center}
        \vspace*{1cm}
            
        \Huge
        \textbf{Manejo eficiente de la memoria}
            
        \vspace{0.8 cm}
        \LARGE
        Taller Memoria Informática II
            
        \vspace{1.5cm}
            
        \textbf{Juan Pablo Guerra Callejas}
        
        \vspace{0.8 cm}
        
        \textbf{Docentes: Augusto Salazar\\
                Jonathan Gómez}
        \vfill
            
        \vspace{0.8cm}
            
        \Large
        Despartamento de Ingeniería Electrónica y Telecomunicaciones\\
        Universidad de Antioquia\\
        Medellín\\
        Septiembre de 2020
            
    \end{center}
\end{titlepage}

\tableofcontents
\newpage
\section{Funcionalidad de la memoria en un computador}\label{intro}


El avance tecnológico ha ido fuertemente ligado a la evolución de los dispositivos electrónicos, un ejemplo de estos dispositivos es el computador. El computador es una máquina digital programable con la que el usuario interactúa utilizando una serie de comandos para procesar datos de entrada con el fin de obtener un resultado o información a la salida. La mayoría de las personas que se relacionan con el mundo de la tecnología son capaces de discernir entre el software y el hardware del computador, lo realmente complicado es profundizar en las partes más importante que lo componen. Entre estas partes se encuentra un componente que parece conceptualmente sencillo, pero abarca la mayor diversidad de tipos, tecnología y coste en comparación con los demás componentes del computador. \\

Tal y como se definió el computador es posible hacerse una idea del papel tan primordial que presenta la memoria. Esta, es el dispositivo donde se almacena temporalmente toda la información con la que trabajan los microprocesadores. Se podría decir que la memoria es un intermediario entre la información almacenada en el disco duro y el procesamiento de información que realizan los microprocesadores. Si bien es cierto que el procesador puede utilizar el disco duro como la fuente directa para el procesado, este sería muchísimo más lento, debido principalmente a la distancia que se encuentra del núcleo del microprocesador.
Una vez el sistema deja de utilizar el contenido extraído del disco duro y almacenado en celdas de memoria más cercanas, se elimina de dichos espacios, de este modo es posible utilizar la capacidad de la memoria en otras instrucciones del procesador y así priorizar la velocidad de ejecución.\\

En resumen, la memoria se utiliza principalmente para agilizar el procesamiento de información en un computador, almacenando el contenido extraído del disco duro con la ayuda del microprocesador hacia un espacio o dirección de memoria más cercano a los núcleos.


\section{Tipos de memoria conocidos} \label{contenido}


\subsection{Memoria Caché (L1, L2, L3)}
\subsection{Memoria RAM (Random Acces Memory)}
\subsection{Memoria DRAM (Dynamic RAM)}
\subsection{Memoria SRAM (Static RAM)}
\subsection{Memoria virtual}
\subsection{Disco duro}


\section{Gestión de memoria}

\section{Conclusiones}


\bibliographystyle{IEEEtran}
\bibliography{references}

\end{document}
