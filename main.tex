\documentclass{article}
\usepackage[utf8]{inputenc}
\usepackage[spanish]{babel}
\usepackage{listings}
\usepackage{graphicx}
\graphicspath{ {images/} }
\usepackage{cite}

\begin{document}

\begin{titlepage}
    \begin{center}
        \vspace*{1cm}
            
        \Huge
        \textbf{Manejo eficiente de la memoria}
            
        \vspace{0.8 cm}
        \LARGE
        Taller Memoria Informática II
            
        \vspace{1.5cm}
            
        \textbf{Juan Pablo Guerra Callejas}
        
        \vspace{0.8 cm}
        
        \textbf{Docentes: Augusto Salazar\\
                Jonathan Gómez}
        \vfill
            
        \vspace{0.8cm}
            
        \Large
        Despartamento de Ingeniería Electrónica y Telecomunicaciones\\
        Universidad de Antioquia\\
        Medellín\\
        Septiembre de 2020
            
    \end{center}
\end{titlepage}

\tableofcontents
\newpage
\section{Funcionalidad de la memoria en un computador}\label{intro}


\section{Tipos de memoria conocidos} \label{contenido}
Esta sección es para ver qué pasa con los comandos que definen texto.
\subsection{Citación}
Vamos a citar por ejemplo un artículo de \textbf{Albert Einstein} \cite{einstein}.
También es posible citar libros \cite{dirac} o documentos en línea \cite{knuthwebsite}.\\\\
Revisar en la última sección el formato de las referencias en IEEE.

\subsection{Incluir código en el documento}
%
A continuación, se presenta el código \ref{codigo_ejemplo}, que nos permite incluir en el informe partes de código que requieran una explicación exhaustiva.
\begin{lstlisting}[language=C++, caption=Ejemplo, label=codigo_ejemplo]
#include <stdio.h>
#define N 10
/* Block
 * comment */

int main()
{
    int i;

    // Line comment.
    puts("Hello world!");
    
    for (i = 0; i < N; i++)
    {
        puts("LaTeX is also great for programmers!");
    }

    return 0;
}
\end{lstlisting}
En la sección \ref{imagenes}, se presentará como añadir ilustraciones al texto.

\section{Inclusión de imágenes} \label{imagenes}

En la Figura (\ref{fig:cpplogo}), se presenta el logo de C++ contenido en la carpeta images.

\begin{figure}[h]
\includegraphics[width=4cm]{cpplogo.png}
\centering
\caption{Logo de C++}
\label{fig:cpplogo}
\end{figure}

Las secciones (\ref{intro}), (\ref{contenido}) y (\ref{imagenes}) dependen del estilo del documento.

\bibliographystyle{IEEEtran}
\bibliography{references}

\end{document}
